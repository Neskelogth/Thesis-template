% !TEX encoding = UTF-8
% !TEX TS-program = pdflatex
% !TEX root = ../tesi.tex

%**************************************************************
\chapter{Introduction}
\label{cap:introduzione}
%**************************************************************

	In the last few decades, video games became more important than anyone would have imagined. 
	As of August 2021, there are more than 3.2 billion gamers on Earth \cite{site:statista}. 
	That means about 40\% of the population plays video games. 
	In the video games panorama, the online portion is vastly widespread. 
	Some examples, taken from the most popular online games in 2021 are:
	League of Legends (LOL), 
	PlayerUnknown's Battlegrounds(PUBG), 
	Fortnite,
	Counter-Strike: Global Offensive (CS:GO), 
	Hearthstone and 
	Defense of the Ancients 2 (DOTA 2) \cite{site:firstsportz}.
	The only common thing between these games is their online nature. 
	The games range in a variety of genres, for example
	card games like Hearthstone, 
	\gls{FPS} (FPS) like CS:GO, 
	Multiplayer Online Battle Arena (MOBA) like DOTA 2 and LOL, 
	Battle Royale like PUBG, or
	Third-Person Shooter (TPS) like Fortnite.
	
	The fact that people play online opens the way to harmful behaviours because of the interactions with strangers. 
	There is a lot of people interested in stealing personal data from others. 
	All major video game publishers take countermeasures, such as \gls{Ban}, to mitigate the problem. 
	Banning a user, though, does not prevent the same person from starting a new account. 
	A company has no way to understand if a user that created a new profile was banned before. 
	To resolve this issue, in PvP: Profiling versus Player \cite{10.1007/978-3-030-62974-8_22} the authors proved that it is possible to identify a player by exploiting their play style. 
	The neural network the authors of the paper realized reached 96\% accuracy in recognizing a given player by the play style in DOTA 2.
	So the recognizing system proved to be working on DOTA 2, but there was no evidence of the fact that works also on other video games.
	Given the difference of each video game, it is hard to identify some play style elements that are common through different games. 
	In this work, the goal is to prove that is possible to determine the identity of a given player in Counter-Strike: Global Offensive. 
	Proving that there is the possibility to identify a player on CS:GO would be very important because it would prove that there is the possibility to recognize players in different genres of video games. 
	The two games, DOTA 2 and CS:GO, differ from each other in many aspects:
	the type of game,
	the objectives to carry out through the match, and
	the vision that the player has
	are just some of the differences.
	The work tries to find some features that could be important to understand the play style. 
	This process, starting from the logs of the parsed matches, identified some features that can represent, in CS:GO, the play style of a given person.
	Some example of feature, in a \gls{FPS} that could be part of a unique play style are:
	the movement of the mouse, 
	the way of moving in the map, 
	the number of jumps, 
	the number of all the actions in the game, 
	the number of kills, 
	and the number of assists.\\
	We started with the logs of the parsed matches and preprocessed them to be used in a machine learning project. 
	We reproduced the model used for the DOTA 2 project and used it to predict the player given the match. 
	We trained the model using 50 players and 100 games each. 
	As a result, we got a model that scored 92\% accuracy in the validation set.
	The conclusion we can draw from this study is that the play style can be considered biometric also in CS:GO, and possibly in other video games, as long as there are features to represent it.
