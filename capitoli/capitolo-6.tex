% !TEX encoding = UTF-8
% !TEX TS-program = pdflatex
% !TEX root = ../tesi.tex

%**************************************************************
\chapter{Discussion}
\label{cap:results}
%**************************************************************

%\intro{In this chapter there is a discussion about our results and how their effects.}\\

%**************************************************************

Being able to recognize a player's play style leads to a lot of different discussions. \\
We firmly believe that this capability is helpful to decrease the number of \gls{Smurf}, \gls{Booster}, and account sellers. 
These three problems are common to the vast majority of online video games. 
In the first case, \gls{Smurf} ruin the experience for the inexperienced players that play with them. 
In the case of \gls{Booster}, there is the opposite problem, since the boosted player is not skilled enough to be on par with the other players he plays with and ruins their match as a consequence. 
Moreover, all the games played by the booster player are ruined, similarly to smurfing. 
Selling an account is problematic because it can result in smurfing or boosting problems.
The capability to punish these people and possibly ban every account they play with is a repellent for the problem. 
The same logic applies to scammers and, more generically, to every form of bad behaviour.
%The DOTA 2 paper, which we took inspiration from, already discussed these possibilities in greater detail. 
Our work proved that there are online video games, a part from DOTA 2, whose players can be identified. 
This opens the door to new studies to see if the play style can be considered biometric also in other video games. 
If even other games offer this possibility, video game publishers can follow this possibility to punish players that commit severe violations of the rules.\\
On the other side, though, a system like this seriously menaces the players' privacy. 
Since some games let a user log in through social media, it is ideally possible to use transfer learning to track gamers among different games and, 
if they logged in through the social media method, take all the public information from the social media used for the authentication.
