% !TEX encoding = UTF-8
% !TEX TS-program = pdflatex
% !TEX root = ../tesi.tex

%**************************************************************
% Sommario
%**************************************************************
\cleardoublepage
\phantomsection
\pdfbookmark{Abstract}{Abstract}
\begingroup
\let\clearpage\relax
\let\cleardoublepage\relax
\let\cleardoublepage\relax

\chapter*{Abstract}

	Since their creation, there has been an increasing interest in video games, especially online video games. 
	Harmful behaviours are a growing phenomenon in online communities and video games. 
	Many people want to trick others for their own benefit.
	In online video games, players may be adults, teenagers or even children and could be easy to trick them into giving some sensitive data that shouldn't be shared.
	This is only an example of harmful behaviours, there are many other possibilities, such as stealing an account or hack into a player's profile to steal pieces of information.
	This issue, addressed by all major video game publishers, is difficult to undertake.
	In 2020 a paper suggested that it could be possible to recognize players exploiting game data (such as parsed replays of the matches). 
	It showed how, in DOTA2, there was the possibility to identify a given player. 
	This thesis' purpose is to demonstrate that this is possible even in a different game, Counter-Strike: Global Offensive. 
	The work done proves that it's possible to identify a player out of 50, using 100 matches each as training, with a 92\% accuracy.

\endgroup			

\vfill

